\hypertarget{index_problem}{}\section{Problem}\label{index_problem}
Develop an interrupt driven software driver for a 2x24 L\-C\-D with an onboard K\-S0066\-U dot matrix controller. This driver is to be used with the latest build of Free\-R\-T\-O\-S. It should allow for the direct writing of data to the L\-C\-D.\hypertarget{index_design}{}\subsection{Design and Decomposition}\label{index_design}
The drivers for the dot matrix controller will be able to write to both lines of the L\-C\-D display. This will be done by implementing A\-P\-Is that can be called from the latest Free\-R\-T\-O\-S O\-S distribution. All driver definitions, functions, and configurations will be written following Free\-R\-T\-O\-S naming conventions for was of use. The A\-P\-Is will give instructions to the dot matrix controller. These instructions will be available through A\-P\-I calls. The available instructions to call will include turning the display on and off, initializing the display, clearing the display, change the entry mode of the display, change the cursor mode, change the cursor position, write a string to the display, and write a variable to the display. An A\-P\-I call will also be available to pass custom instructions or data to the dot matrix controller.\hypertarget{index_testing}{}\subsection{Testing}\label{index_testing}
To test the L\-C\-D display drivers, a working Free\-R\-T\-O\-S build will be used to create a single task. This task will be used to host each A\-P\-I call individually to test for proper usage from each A\-P\-I call. A full system test will be done by using every A\-P\-I call in the task, followed by a write command to the L\-C\-D to ensure each instruction works. This will ensure each instruction works properly with the other instructions in the driver.\hypertarget{index_documentation}{}\subsection{Documentation}\label{index_documentation}
Doxygen will be used to provide detailed documentation of all code done for the L\-C\-D driver. Each code file, function, and variable will have proper documentation, stating the purpose of the item being documented, both briefly and in detail. As well, each code file and function will have a revision history. All code will contain useful comments in-\/line using Doxygen hooks. The end result shall be useful documentation that is easy to follow. This final documentation shall include call and caller graphs so flow within the code file can be understood. Any file that does not pertain to the L\-C\-D driver will not be included or documented within doxygen, including Free\-R\-T\-O\-S code.\hypertarget{index_version}{}\subsection{Version Control}\label{index_version}
Version Control will be implemented using Git. Working in a group of four requires the use of a distributed version control system. Git has the feature set to clone, edit, and push changes back to the working code. All changes will receive their own commit, with the text of the commit being brief information as to what was changed. The main working repository will be hosted on a cloud-\/based sharing service, such as Git\-Hub. This will allow for all persons editing the code to be able to clone and work from the most recent revision of the code.

Using Git\-Hub will allow for easy hosting of the remote Git repository. Due to the fact that many of the people in our group are limited by the network settings and securities of apartment complexes, this will work nicely. Assigning one person to be the integration manager, all other members of the group will work as developers. The manager is the person who originally uploads the code, and therefore has the responsibility to merge changes into that repository. The developers clone the manager’s repository to make changes, and notify the manager when they want their changes merged with the blessed repository. Every member of the group will get the chance to be the manager, giving them the experience to do a merge. 